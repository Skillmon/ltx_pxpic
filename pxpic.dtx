% \iffalse meta-comment
%
% File: pxpic.dtx Copyright (C) 2021 Jonathan P. Spratte
%
% This work  may be  distributed and/or  modified under  the conditions  of the
% LaTeX Project Public License (LPPL),  either version 1.3c  of this license or
% (at your option) any later version.  The latest version of this license is in
% the file:
%
%   http://www.latex-project.org/lppl.txt
%
% ------------------------------------------------------------------------------
%
%<*driver>^^A>>=
\def\nameofplainTeX{plain}
\ifx\fmtname\nameofplainTeX\else
  \expandafter\begingroup
\fi
\input l3docstrip.tex
\askforoverwritefalse
\preamble

--------------------------------------------------------------
pxpic -- draw pixel pictures
E-mail: jspratte@yahoo.de
Released under the LaTeX Project Public License v1.3c or later
See http://www.latex-project.org/lppl.txt
--------------------------------------------------------------

Copyright (C) 2021 Jonathan P. Spratte

This  work may be  distributed and/or  modified under  the conditions  of the
LaTeX Project Public License (LPPL),  either version 1.3c  of this license or
(at your option) any later version.  The latest version of this license is in
the file:

  http://www.latex-project.org/lppl.txt

This work is "maintained" (as per LPPL maintenance status) by
  Jonathan P. Spratte.

This work consists of the file  pxpic.dtx
and the derived files           pxpic.pdf
                                pxpic.sty

\endpreamble
% stop docstrip adding \endinput
\postamble
\endpostamble
\generate{\file{pxpic.sty}{\from{pxpic.dtx}{pkg}}}
\ifx\fmtname\nameofplainTeX
  \expandafter\endbatchfile
\else
  \expandafter\endgroup
\fi
%
\ProvidesFile{pxpic.dtx}[2021-01-10 v1.0 draw pixel pictures]
\PassOptionsToPackage{full}{textcomp}
\documentclass{l3doc}
\RequirePackage[oldstylenums,nott]{kpfonts}
\usepackage{pxpic}
\input{glyphtounicode}
\pdfgentounicode=1
\RequirePackage{listings}
\RequirePackage{booktabs}
\RequirePackage{array}
\RequirePackage{collcell}
\RequirePackage{xcolor}
\RequirePackage{caption}
\RequirePackage{microtype}
\RequirePackage{accsupp}
\RequirePackage{enumitem}
\lstset
  {
    ,flexiblecolumns=false
    ,basewidth=.53em
    ,gobble=2
    ,basicstyle=\fontfamily{jkp}\itshape
    ,morekeywords=^^A
      {^^A
        \pxpic,\pxpicsetup
      }
    ,morecomment=[l]\%
    ,commentstyle=\color[gray]{0.4}
    ,literate={\{}{{\CodeSymbol\{}}{1}
              {\}}{{\CodeSymbol\}}}{1}
    ^^A,literate=*{<key>}{\key}{4}{<set>}{\set}{4}
  }
\newcommand*\CodeSymbol[1]{\textbf{#1}}
\RequirePackage{randtext}
\let\metaORIG\meta
\protected\def\meta #1{\texttt{\metaORIG{#1}}}
\renewcommand*\thefootnote{\fnsymbol{footnote}}
\definecolor{pxpicred}{HTML}{9F393D}
\colorlet{pxpicgrey}{black!75}
\makeatletter
\newcommand*\pxpicname
  {^^A
    \texorpdfstring
      {^^A
        \mbox
          {^^A
            \BeginAccSupp{ActualText=pxpic}^^A
            \href{https://github.com/Skillmon/ltx_pxpic}{\pxpiclogo}^^A
            \EndAccSupp{}^^A
          }^^A
      }
      {pxpic}^^A
  }
\definecolor{expkvred}{HTML}{9F393D}
\colorlet{expkvgrey}{black!75}
\newcommand*\expkv
  {^^A
    \texorpdfstring
      {^^A
        \mbox
          {^^A
            \BeginAccSupp{ActualText=expkv}^^A
            \href{https://github.com/Skillmon/tex_expkv}
              {^^A
                \rmfamily
                \bfseries
                {\color{expkvgrey}e\kern-.05em x\kern-.05em}^^A
                \lower.493ex
                  \hbox{{\color{expkvgrey}P}\kern-.1em{\color{expkvred}k}}^^A
                \kern-.18em{\color{expkvred}v}^^A
              }^^A
            \EndAccSupp{}^^A
          }^^A
      }
      {expkv}^^A
  }
\hypersetup{linkcolor=red!80!black,urlcolor=purple!80!black}
\DoNotIndex{\baselineskip,\begingroup,\bgroup}
\DoNotIndex{\csname}
\DoNotIndex{\def,\detokenize,\dimexpr}
\DoNotIndex{\egroup,\ekvdef,\ekvdefNoVal,\ekvifdefinedNoVal,\ekvletkv,\ekvparse}
\DoNotIndex{\ekvsetdef,\endcsname,\endgroup,\expandafter}
\DoNotIndex{\hbox,\hskip}
\DoNotIndex{\leavevmode,\let,\long,\lower}
\DoNotIndex{\newcommand,\newdimen}
\DoNotIndex{\PackageError,\protected,\ProvidesPackage}
\DoNotIndex{\relax,\RequirePackage}
\DoNotIndex{\@depth,\@height,\@ifdefinable,\@ifnextchar,\@ifundefined,\@width}
\DoNotIndex{\ekv@ifempty,\ekv@name,\z@}
\DoNotIndex{\color}
\DoNotIndex{\vbox,\vrule}
\@ifdefinable\gobbledocstriptag{\def\gobbledocstriptag#1>{}}
\makeatother
\newenvironment{options}[1][]
  {%
    \begin{description}
      [
        style=nextline
        ,font=\normalfont\ttfamily
        ,labelindent=-.5\marginparwidth
        ,labelwidth=\dimexpr.5\marginparwidth-5pt\relax
        ,labelsep*=5pt
        ,leftmargin=!
        ,#1
      ]%
    \renewenvironment{options}[1][]
      {%
        \begin{description}
          [
            style=nextline
            ,font=\normalfont\ttfamily
            ,#1
          ]%
      }
      {\end{description}}%
  }
  {\end{description}}
\begin{document}
  \title
    {^^A
      \texorpdfstring
        {^^A
          \Huge
          \mbox
            {^^A
              \BeginAccSupp{ActualText=pxpic}^^A
              \href{https://github.com/Skillmon/ltx_pxpic}{\pxpiclogo}^^A
              \EndAccSupp{}^^A
            }^^A
          \\[\medskipamount]
          \Large draw pixel pictures^^A
        }{pxpic - draw pixel pictures}^^A
    }
  \date{2021-01-10 v1.0}
  \author{Jonathan P. Spratte\thanks{\protect\randomize{jspratte@yahoo.de}}}
  \DocInput{pxpic.dtx}
\end{document}
%</driver>^^A=<<
% \fi
%
% \maketitle
% \renewcommand*\thefootnote{\arabic{footnote}}
%
% \begin{abstract}
% \noindent\parfillskip=0pt
% With \pxpicname\ you draw pictures pixel by pixels. It was inspired by a
% \href{https://tex.stackexchange.com/a/63759/117050}
%      {lovely post by Paulo Cereda}
% among other things (most notably a beautiful duck) showcasing the use of
% characters from the Mario video games by Nintendo in \LaTeX.
% \end{abstract}
%
% \tableofcontents
%
% \begin{documentation}^^A>>=
%
% \section{Documentation}
%
% \subsection{Drawing pictures}
%
% \pxpicname\ supports different input modes, all of them have the same basic
% parsing behaviour. A \meta{pixel list} contains the pixel colours. The image
% is built line wise from top left to bottom right. Each row of pixels should be
% a single \TeX\ argument (so either just one token, or a group delimited by
% |{}|), and within each line each pixel in turn should be a single \TeX\
% argument (so either just one token, or a group delimited by |{}|). Spaces and
% hence single newlines in the sources between \meta{pixel list} elements are
% ignored. The different modes are explained in \autoref{sec:modes}. The only
% disallowed token in the \meta{pixel list} is the control sequence
% \cs[no-index]{pxpic@end} (plus the usual restrictions of \TeX\ so no
% unbalanced braces, no macros defined as \cs[no-index]{outer}).
%
% There is a small caveat however: \pxpicname\ draws each pixel individually,
% and there is really no space between them, however some \textsc{pdf} viewers
% fail to display such adjacent lines correctly and leave small gaps (basically
% the same issue which packages like \pkg{colortbl} suffer from as well). In
% print this shouldn't be an issue, but some rasterisation algorithms employed
% by viewers and conversion tools have this deficit.
%
% \begin{function}{\pxpic}
%   \begin{syntax}
%     \cs{pxpic}\oarg{options}\marg{pixel list}
%   \end{syntax}
%   \meta{options} might be any options as listed in \autoref{sec:options}, and
%   \meta{pixel list} is a list of pixels as described above. \cs{pxpic} parses
%   the \meta{pixel list} and draws the corresponding picture. The result is
%   contained in an \cs[no-index]{hbox} and can be used wherever \TeX\ expects
%   an \cs[no-index]{hbox}. As a result, when you're in vertical mode a
%   \cs{pxpic} will form a text line, to prevent this you can use
%   \cs[no-index]{leavevmode} before it. The \cs{pxpic} will be bottom aligned,
%   you can change this using \cs[no-index]{raisebox} (or, if you want, \TeX's
%   \cs[no-index]{raise} and \cs[no-index]{lower} primitives).
% \end{function}
%
% \subsubsection{Examples}
%
% Since the above explanation of the \meta{pixel list} syntax might've been a
% bit cryptic, and a good documentation should contain examples (this doesn't
% claim this documentation is \emph{good}), well, here are some examples (you
% might need to take a look at \autoref{sec:options} and \autoref{sec:modes} to
% fully understand the examples). Examples in this section will use the
% following \cs[no-index]{pxpicsetup}:
%
% \begingroup
% \footnotesize
% \begin{lstlisting}
% \pxpicsetup
%   {
%      mode    = px
%     ,colours = {k=black, r=[HTML]{9F393D}, g=green!75!black, b=[rgb]{0,0,1}}
%     ,skip    = .
%     ,size    = 10pt
%   }
% \end{lstlisting}
% \endgroup
%
% \begingroup
% \pxpicsetup
%   {
%      mode    = px
%     ,colours = {k=black, r=[HTML]{BF292D}, g=green!75!black, b=blue}
%     ,skip    = .
%     ,size    = 10pt
%   }
%
% We can draw a small cross rather easily:\par\nobreak
% \noindent
% \begin{minipage}[c]{.25\linewidth}\footnotesize
% \begin{lstlisting}
% \pxpic
%   {
%     {.k}
%     {kkk}
%     {.k}
%   }
% \end{lstlisting}
% \end{minipage}
% \begin{minipage}[c]{.45\linewidth}
% \leavevmode
% \pxpic
%   {
%     {.k}
%     {kkk}
%     {.k}
%   }
% \end{minipage}
%
% A small multicoloured grid:\par\nobreak
% \noindent
% \begin{minipage}[c]{.25\linewidth}\footnotesize
% \begin{lstlisting}
% \pxpic
%   {
%     {brgk}
%     {kbrg}
%     {gkbr}
%     {rgkb}
%   }
% \end{lstlisting}
% \end{minipage}
% \begin{minipage}[c]{.45\linewidth}
% \leavevmode
% \pxpic
%   {
%     {brgk}
%     {kbrg}
%     {gkbr}
%     {rgkb}
%   }
% \end{minipage}
%
% A heart (shamelessly copied example from \pkg{pixelart}):\par\nobreak
% \noindent
% \begin{minipage}[c]{.25\linewidth}\footnotesize
% \begin{lstlisting}
% \pxpic
%   {
%     {..rr.rr}
%     {.rrrrrrr}
%     {rrrrrrrrr}
%     {rrrrrrrrr}
%     {rrrrrrrrr}
%     {.rrrrrrr}
%     {..rrrrr}
%     {...rrr}
%     {....r}
%   }
% \end{lstlisting}
% \end{minipage}
% \begin{minipage}[c]{.45\linewidth}
% \leavevmode
% \pxpic
%   {
%     {..rr.rr}
%     {.rrrrrrr}
%     {rrrrrrrrr}
%     {rrrrrrrrr}
%     {rrrrrrrrr}
%     {.rrrrrrr}
%     {..rrrrr}
%     {...rrr}
%     {....r}
%   }
% \end{minipage}
%
% Using |mode=rgb| to draw a short coloured line:\par\nobreak
% \noindent
% \begin{minipage}[c]{.5\linewidth}\footnotesize
% \begin{lstlisting}
% \pxpic[mode=rgb]{{{1,0,1}{1,1,0}{0,1,1}}}
% \end{lstlisting}
% \end{minipage}
% \begin{minipage}[c]{.45\linewidth}
% \leavevmode
% \pxpic[mode=rgb]{{{1,0,1}{1,1,0}{0,1,1}}}
% \end{minipage}
%
% A multicoloured grid using skips and |mode=cmy|:\par\nobreak
% \noindent
% \begin{minipage}[c]{.5\linewidth}\footnotesize
% \begin{lstlisting}
% \pxpic[mode=cmy]
%   {
%     {{1,0,1} {1,1,0} {0,1,1} {}     }
%     {{}      {1,0,1} {1,1,0} {0,1,1}}
%     {{0,1,1} {}      {1,0,1} {1,1,0}}
%     {{1,1,0} {0,1,1} {}      {1,0,1}}
%   }
% \end{lstlisting}
% \end{minipage}
% \begin{minipage}[c]{.45\linewidth}
% \leavevmode
% \pxpic[mode=cmy]
%   {
%     {{1,0,1} {1,1,0} {0,1,1} {}     }
%     {{}      {1,0,1} {1,1,0} {0,1,1}}
%     {{0,1,1} {}      {1,0,1} {1,1,0}}
%     {{1,1,0} {0,1,1} {}      {1,0,1}}
%   }
% \end{minipage}
%
% Showing the difference between a skipped and a white pixel:\par\nobreak
% \noindent
% \begin{minipage}[c]{.5\linewidth}\footnotesize
% \begin{lstlisting}
% \pxpicsetup{colours = {w=white}}
% \colorbox{gray}{\pxpic{{bbb}{b.b}{bbb}}}
% \colorbox{gray}{\pxpic{{bbb}{bwb}{bbb}}}
% \end{lstlisting}
% \end{minipage}
% \begin{minipage}[c]{.45\linewidth}
% \pxpicsetup{colours = {w=white}}
% \colorbox{gray}{\pxpic{{bbb}{b.b}{bbb}}}
% \colorbox{gray}{\pxpic{{bbb}{bwb}{bbb}}}
% \end{minipage}
%
% Another example might be the definition of \cs[no-index]{pxpiclogo} in
% \autoref{sec:implementation:pxpiclogo}.
%
% Who still needs \env{picture}-mode or complicated packages like
% \pkg{pstricks} or Ti\emph{k}Z with such pretty pictures?
%
% \endgroup
%
% \subsection{Setting options}\label{sec:options}
%
% To control its behaviour \pxpicname\ uses a key=value interface powered by
% \expkv. Options can be set either in the optional argument of
% \cs[no-index]{pxpic} or with
%
% \begin{function}{\pxpicsetup}
%   \begin{syntax}
%     \cs{pxpicsetup}\marg{options}
%   \end{syntax}
%   Sets the \meta{options} locally to the current \TeX\ group.
% \end{function}
%
% \medskip\noindent
% Package options are not supported.
%
% \bigskip
% The available options are
%
% \begin{options}
%   \item[colors] Define pixel colours for |mode=px|, see \autoref{sec:colours}
%     for a description of the value's syntax. No pixel definitions are made by
%     the package.
%   \item[colours] \emph{see \texttt{colors}.}
%   \item[ht] Set the height of the pixels.
%   \item[mode] Set the used mode, see \autoref{sec:modes} for available modes.
%     Initial value is |px|.
%   \item[size] Set both |ht| and |wd|. Initial value is \the\pxpicHT.
%   \item[skip] Define the value to be a skip (an empty space of width |wd|) in
%     |mode=px|. No skip definitions are made by the package.
%   \item[wd] Set the width of the pixels.
% \end{options}
%
% \subsubsection{Colour syntax}\label{sec:colours}
%
% In the value of the |colours| option you'll have to use the following syntax.
% Use a comma separated key=value list in which each key corresponds to a new
% pixel name for |mode=px|, and each value to the used colour. If the colour
% starts with an opening bracket use the complete value as is behind
% \cs[no-index]{color}, else use the whole value as the first mandatory
% argument to \cs[no-index]{color} with a set of braces added. For example
% to define |r| as the named colour |red|, and |x| as the colour |#abab0f| (in
% the \textsc{html} colour model) use:
% \begin{lstlisting}
% colours = {r=red, x=[HTML]{abab0f}}
% \end{lstlisting}
%
% \subsubsection{Available modes}\label{sec:modes}
%
% \begin{options}
%   \item[px]
%     As already mentioned, \pxpicname\ supports different modes of input. The
%     easiest to use |mode| is |px|, in which each element of the \meta{pixel
%     list} has been previously defined as either a coloured pixel (using the
%     |colour| option) or as a skipped pixel (using the |skip| option, resulting
%     in a fully transparent pixel). This is the initial mode \pxpicname\ uses.
%     But other options are available as well.
%   \item[named]
%     Another |mode| is |named|, in which each element of the \meta{pixel list}
%     should be a named colour (or colour expression) known to \pkg{xcolor}.
%     Each element will be used like so: |{\color{|\meta{element}|}\px}|. An
%     exception is an element which is empty (|{}|), which will be a skipped
%     pixel.
%   \item[rgb, cmy, cmyk, hsb, Hsb, tHsb, gray, RGB, HTML, HSB, Gray, wave]
%     The modes |rgb|, |cmy|, |cmyk|, |hsb|, |Hsb|, |tHsb|, |gray|, |RGB|,
%     |HTML|, |HSB|, |Gray|, and |wave| correspond to the different colour
%     models supported by \pkg{xcolor}. With these modes each element of the
%     \meta{pixel list} will be the values in these colour models, so they'll be
%     used like so: |{\color[|\meta{mode}|]{|\meta{element}|}\px}|. An
%     exception is an element which is empty (|{}|), which will be a skipped
%     pixel.
% \end{options}
%
% You can define additional modes selectable with the |mode| option using the
% macros |\pxpicnewmode| or |\pxpicsetmode|.
%
% \subsection{Other customisation macros}
%
% \begin{function}{\pxpicnewmode,\pxpicsetmode}
%   \begin{syntax}
%     \cs{pxpicnewmode}\marg{name}\marg{definition}
%   \end{syntax}
%   You can define your own modes with \cs{pxpicnewmode}. Inside
%   \meta{definition} |#1| is the currently parsed item in the
%   \cs[no-index]{pxpic} pixel list. You can output a pixel using
%   \cs[no-index]{px}, and skip a pixel using \cs[no-index]{pxskip}. The pixel
%   will use the currently active colour (so if you want to draw a red pixel you
%   could use |{\color{red}\px}|). \cs{pxpicnewmode} will throw an error if
%   you try to define a mode which already exists, \cs{pxpicsetmode} has no
%   checks on the name.
% \end{function}
%
% \begin{function}{\pxpicforget}
%   \begin{syntax}
%     \cs{pxpicforget}\marg{px}
%   \end{syntax}
%   Undefines the \meta{px} definition for use in |mode=px| (or skip symbol)
%   added with the |colours| (or |skip|) option.
% \end{function}
%
% \subsection{Other macros}
%
% \begin{function}{\px,\pxskip}
%   \begin{syntax}
%     \cs{px}
%   \end{syntax}
%   Inside of a \cs[no-index]{pxpic} the macro \cs{px} draws a pixel (of the
%   currently active colour), and \cs{pxskip} leaves out a pixel (so this one
%   pixel is fully transparent). Use this in the \meta{definition} of a mode in
%   \cs[no-index]{pxpicnewmode}.
% \end{function}
%
% \begin{variable}{\pxpicHT,\pxpicWD}
%   These two are |dimen| registers storing the height and width of the pixels.
% \end{variable}
%
% \begin{function}{\pxpiclogo}
%   \begin{syntax}
%     \cs{pxpiclogo}\oarg{size}
%   \end{syntax}
%   This draws the logo of \pxpicname. The \meta{size} controls the pixel size.
% \end{function}
%
% \subsection{Miscellaneous}
%
% If you find bugs or have suggestions I'll be glad to hear about it, you can
% either open a ticket on Github (\url{https://github.com/Skillmon/ltx_pxpic})
% or email me (see the first page).
%
%
% \end{documentation}^^A=<<
%
% \begin{implementation}^^A>>=
%
% \gobbledocstriptag
%<*pkg>
%
% \clearpage
%
% \section{Implementation}^^A>>=
%
% Report who we are
%    \begin{macrocode}
\ProvidesPackage{pxpic}[2021-01-10 v1.0 draw pixel pictures]
%    \end{macrocode}
% and load dependencies
%    \begin{macrocode}
\RequirePackage{xcolor}
\RequirePackage{expkv}
%    \end{macrocode}
% \begin{variable}{\pxpicHT,\pxpicWD}
%   These two variables store the height and width of a pixel.
%    \begin{macrocode}
\@ifdefinable\pxpicHT{\newdimen\pxpicHT}
\@ifdefinable\pxpicWD{\newdimen\pxpicWD}
\pxpicHT=1pt
\pxpicWD=\pxpicHT
%    \end{macrocode}
% \end{variable}
%
%
% \subsection{Options}
%
%    \begin{macrocode}
\protected\ekvdef{pxpic}{size}
  {\pxpicHT=\dimexpr#1\relax\pxpicWD=\dimexpr#1\relax}
\protected\ekvdef{pxpic}{ht}{\pxpicHT=\dimexpr#1\relax}
\protected\ekvdef{pxpic}{wd}{\pxpicWD=\dimexpr#1\relax}
\protected\ekvdef{pxpic}{colors}{\ekvparse\pxpic@noval\pxpic@setcolor{#1}}
\ekvletkv{pxpic}{colours}{pxpic}{colors}
\protected\ekvdef{pxpic}{skip}{\ekvdefNoVal{pxpic@px}{#1}{\pxskip}}
\protected\ekvdef{pxpic}{mode}
  {%
    \@ifundefined{pxpic@parse@px@#1}%
      {\pxpic@unknown@mode{#1}}%
      {%
        \expandafter\let\expandafter\pxpic@parse@px
          \csname pxpic@parse@px@#1\endcsname
      }%
  }
%    \end{macrocode}
%
%
% \subsection{User macros}\label{sec:implementation:pxpiclogo}
%
% \begin{macro}{\pxpic}
% \begin{macro}[internal]{\pxpic@}
%    \begin{macrocode}
\newcommand*\pxpic{\hbox\bgroup\pxpic@}
\newcommand\pxpic@[2][]
  {%
    \vbox
      {%
        \pxpicsetup{#1}%
        \let\px\pxpic@px
        \let\pxskip\pxpic@skip
        \baselineskip\pxpicHT
        \pxpic@parse#2\pxpic@end
      }%
    \egroup
  }
%    \end{macrocode}
% \end{macro}
% \end{macro}
%
% \begin{macro}{\pxpicsetup}
%    \begin{macrocode}
\ekvsetdef\pxpicsetup{pxpic}
%    \end{macrocode}
% \end{macro}
%
% \begin{macro}{\pxpiclogo}
%    \begin{macrocode}
\newcommand*\pxpiclogo[1][.13ex]
  {%
    \begingroup
      \pxpicHT\dimexpr#1\relax
      \pxpicWD\pxpicHT
      \leavevmode
      \lower3.2\pxpicHT\pxpic
        [mode=px,colours={o=[HTML]{9F393D},g=black!75},skip=.]
        {
          {............................................g}
          {...........................................gggg}
          {.oooo.......................gggg...........ggg}
          {.ooooo...oo......oo...oo....ggggg...gg......g..........g}
          {.ooooooooooo...ooooo..oooo..ggggggggggg...ggggg...ggggggg}
          {..ooooo..oooo.ooooooooooo....ggggg..gggg.ggggggg.ggggggggg}
          {...oooo..oooo....ooooo........gggg..gggg...gggg..gggg.ggg}
          {...oooo..oooo.....oooo........gggg..gggg...gggg..gggg}
          {.oooooo..oooo.....ooooo.....gggggg..gggg...gggg..gggg}
          {oooooooooooo...ooooooooo...gggggggggggg....gggg..ggggggggg}
          {o.oooooooo....ooooo.oooooo.g.gggggggg......ggggg..ggggggg}
          {...ooo.o......o.oo...oo.......ggg.g.........gg......ggg}
          {...ooo........................ggg}
          {...ooo........................ggg}
          {....o..........................g}
        }%
    \endgroup
  }
%    \end{macrocode}
% \end{macro}
%
% \begin{macro}{\pxpicforget}
%    \begin{macrocode}
\newcommand\pxpicforget[1]
  {\expandafter\let\csname\ekv@name{pxpic@px}{#1}N\endcsname\pxpic@undef}
%    \end{macrocode}
% \end{macro}
%
% \begin{macro}{\pxpicnewmode,\pxpicsetmode}
%    \begin{macrocode}
\protected\def\pxpicnewmode#1#2%
  {\expandafter\newcommand\csname pxpic@parse@px@#1\endcsname[1]{#2}}
\protected\def\pxpicsetmode#1#2%
  {\long\expandafter\def\csname pxpic@parse@px@#1\endcsname##1{#2}}
%    \end{macrocode}
% \end{macro}
%
%
% \subsection{Parser}
%
% \begin{macro}[internal]{\pxpic@ifend,\pxpic@ifempty}
%    \begin{macrocode}
\long\def\pxpic@ifend#1\pxpic@end{}
\let\pxpic@ifempty\ekv@ifempty
%    \end{macrocode}
% \end{macro}
%
% \begin{macro}[internal]{\pxpic@parse,\pxpic@done}
%    \begin{macrocode}
\newcommand\pxpic@parse[1]
  {%
    \pxpic@ifend#1\pxpic@done\pxpic@end
    \hbox{\pxpic@parseline#1\pxpic@end}%
    \pxpic@parse
  }
\long\def\pxpic@done\pxpic@end\hbox#1\pxpic@parse{}
%    \end{macrocode}
% \end{macro}
%
% \begin{macro}[internal]{\pxpic@parseline,\pxpic@linedone}
%    \begin{macrocode}
\newcommand\pxpic@parseline[1]
  {%
    \pxpic@ifend#1\pxpic@linedone\pxpic@end
    \pxpic@parse@px{#1}%
    \pxpic@parseline
  }
\long\def\pxpic@linedone\pxpic@end\pxpic@parse@px#1\pxpic@parseline{}
%    \end{macrocode}
% \end{macro}
%
%
% \subsection{Modes}
%
% \begin{macro}[internal]{\pxpic@parse@px@px,\pxpic@parse@px}
%    \begin{macrocode}
\newcommand\pxpic@parse@px@px[1]
  {%
    \ekvifdefinedNoVal{pxpic@px}{#1}
      {\csname\ekv@name{pxpic@px}{#1}N\endcsname}%
      {%
        \pxpic@unknown@px{#1}%
        \pxskip
      }%
  }
\let\pxpic@parse@px\pxpic@parse@px@px
%    \end{macrocode}
% \end{macro}
%
% \begin{macro}[internal]{\pxpic@parse@px@named}
%    \begin{macrocode}
\newcommand\pxpic@parse@px@named[1]
  {%
    \pxpic@ifempty{#1}
      {\pxskip}
      {{\color{#1}\px}}%
  }
%    \end{macrocode}
% \end{macro}
%
% \begin{macro}[internal]
%   {
%     \pxpic@parse@px@rgb,
%     \pxpic@parse@px@cmy,
%     \pxpic@parse@px@cmyk,
%     \pxpic@parse@px@hsb,
%     \pxpic@parse@px@Hsb,
%     \pxpic@parse@px@tHsb,
%     \pxpic@parse@px@gray,
%     \pxpic@parse@px@RGB,
%     \pxpic@parse@px@HTML,
%     \pxpic@parse@px@HSB,
%     \pxpic@parse@px@Gray,
%     \pxpic@parse@px@wave
%   }
%    \begin{macrocode}
\def\pxpic@tmp#1%
  {%
    \pxpicnewmode{#1}%
      {%
        \pxpic@ifempty{##1}
          {\pxskip}
          {{\color[#1]{##1}\px}}%
      }%
  }
\pxpic@tmp{rgb}
\pxpic@tmp{cmy}
\pxpic@tmp{cmyk}
\pxpic@tmp{hsb}
\pxpic@tmp{Hsb}
\pxpic@tmp{tHsb}
\pxpic@tmp{gray}
\pxpic@tmp{RGB}
\pxpic@tmp{HTML}
\pxpic@tmp{HSB}
\pxpic@tmp{Gray}
\pxpic@tmp{wave}
\let\pxpic@tmp\pxpic@undef
%    \end{macrocode}
% \end{macro}
%
%
% \subsection{Pixel and Skip}
%
% \begin{macro}[internal]{\pxpic@px,\pxpic@skip}
%    \begin{macrocode}
\newcommand\pxpic@px{\vrule\@height\pxpicHT\@width\pxpicWD\@depth\z@}
\newcommand\pxpic@skip{\hskip\pxpicWD}
%    \end{macrocode}
% \end{macro}
%
%
% \subsection{Parser for colours}
%
% \begin{macro}[internal]{\pxpic@setcolor,\pxpic@setcolor@a,\pxpic@setcolor@b}
%    \begin{macrocode}
\newcommand\pxpic@setcolor[2]
  {%
    \@ifnextchar[\pxpic@setcolor@a\pxpic@setcolor@b
    #2]\pxpic@end
    {#1}{#2}%
  }
\long\def\pxpic@setcolor@a#1\pxpic@end#2#3%
  {\ekvdefNoVal{pxpic@px}{#2}{{\color#3\px}}}
\long\def\pxpic@setcolor@b#1\pxpic@end#2#3%
  {\ekvdefNoVal{pxpic@px}{#2}{{\color{#3}\px}}}
%    \end{macrocode}
% \end{macro}
%
%
% \subsection{Messages}
%
% \begin{macro}[internal]{\pxpic@noval,\pxpic@unknown@px,\pxpic@unknown@mode}
%   These are just some macros throwing errors, nothing special here.
%    \begin{macrocode}
\newcommand\pxpic@noval[1]
  {\PackageError{pxpic}{Missing colour definition for name `\detokenize{#1}'}{}}
\newcommand\pxpic@unknown@px[1]
  {\PackageError{pxpic}{Unknown pixel `\detokenize{#1}'. Skipping}{}}
\newcommand\pxpic@unknown@mode[1]
  {\PackageError{pxpic}{Unknown mode `#1'}{}}
%    \end{macrocode}
% \end{macro}
%
%^^A=<<
%
% \gobbledocstriptag
%</pkg>
%
% \end{implementation}^^A=<<
%
% \clearpage
% \PrintIndex
%
\endinput
%
^^A vim: ft=tex fdm=marker fmr=>>=,=<<
